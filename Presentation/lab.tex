\documentclass[compress,11pt,xcolor=svgnames,aspectratio=169]{beamer}
\usetheme{Esiwace}
\usefonttheme[onlysmall]{structurebold}

%\usepackage{caption}
%\captionsetup{labelformat=empty, format=plain, labelsep=none,textfont=footnotesize}

\usepackage[utf8]{inputenc}
\usepackage[english]{babel}
\usepackage[T1]{fontenc}
\usepackage{eurosym}
\usepackage{ulem}
\usepackage{listings}
\usepackage{ragged2e}  %For \justify and \justifying, but it's not working
\usepackage{pgfgantt}
\usepackage{comment}
\usepackage{verbatim}
\usepackage{varwidth}

\newcommand{\lr}[1]{\textcolor{cyan}{LR: #1}}

\makeatletter
\renewcommand{\itemize}[1][]{%
  \beamer@ifempty{#1}{}{\def\beamer@defaultospec{#1}}%
  \ifnum \@itemdepth >2\relax\@toodeep\else
    \advance\@itemdepth\@ne
    \beamer@computepref\@itemdepth% sets \beameritemnestingprefix
    \usebeamerfont{itemize/enumerate \beameritemnestingprefix body}%
    \usebeamercolor[fg]{itemize/enumerate \beameritemnestingprefix body}%
    \usebeamertemplate{itemize/enumerate \beameritemnestingprefix body begin}%
    \list
      {\usebeamertemplate{itemize \beameritemnestingprefix item}}
      {\def\makelabel##1{%
          {%
            \hss\llap{{%
                \usebeamerfont*{itemize \beameritemnestingprefix item}%
                \usebeamercolor[fg]{itemize \beameritemnestingprefix item}##1}}%
          }%
        }%
      }
  \fi%
  \beamer@cramped%
  \justifying% NEW
  %\raggedright% ORIGINAL
  \beamer@firstlineitemizeunskip%
}
\makeatother

\tcbuselibrary{raster}
\newcommand{\sectionIntroHidden}{
\begin{frame}{Outline}
  \tableofcontents[currentsection,subsectionstyle=hide/hide/hide]
\end{frame}
}

\DeclareGraphicsExtensions{.pdf,.png,.jpg,}
\graphicspath{{./fig/}}

\title[Input/Output and Middleware -- Lab Session]{2020 Summer School on Effective HPC for Climate and Weather \\[0.5cm] Input/Output and Middleware}
\author[Pedro, Kunkel]{Luciana Pedro, Julian Kunkel
}
\institute[WP4 Team]{Department of Computer Science, University of Reading}
\date{25 August 2020}

\begin{document}

%%%%%%%%%%%%%%%%%%%%%%%%%%%%%%%%%%%%%%%%%%%%%%%%%%%%%%%%%%%%
\begin{frame}[plain]
    \titlepage
\end{frame}

%%%%%%%%%%%%%%%%%%%%%%%%%%%%%%%%%%%%%%%%%%%%%%%%%%%%%%%%%%%%
\begin{withoutheadline}
\begin{frame}{Outline}
    \begin{centering}
    \tableofcontents[hideallsubsections]
    \end{centering}

    \disclaimer
\end{frame}
\end{withoutheadline}

% Input/Output and Middleware
% Climate and weather research is typically data-intensive and applications must utilise input/output efficiently. Often, a user struggles to assess observed performance leading to superflux attempts to tune the application and optimise performance in a wrong layer of the stack. The content of this session is twofold. Firstly, we discuss storage layers focusing on the NetCDF middleware and provide a performance model that aids users to identify inefficient I/O. Secondly, we introduce the NetCDF Climate and Forecast (CF) conventions that are often used as a standard to exchange data.

\begin{frame}[fragile]{Learning Objectives}

\begin{itemize}
\setlength\itemsep{1cm}
  \item Execute programs in C that read and write NetCDF files in a metadata-aware manner
  \item Analyse, manipulate and visualise NetCDF data
%  \item Implement an application that utilizes parallel I/O to store and analyze data
\end{itemize}

\end{frame}

\section{NetCDF Files and C}

% Inserting [fragile] to be able to work with \verb

\subsection{Introduction}

\begin{frame}[fragile]{References}

\begin{itemize}
\setlength\itemsep{0.6cm}

  \item The files and data used in this presentation were collected on the Unidata website.

    \begin{itemize}
    \item \url{https://www.unidata.ucar.edu/}
    \end{itemize}

  \item All files used here are available in the following Git Repository:

    \begin{itemize}
    \item \url{https://github.com/ESiWACE/io-training}
    \end{itemize}

  \item These files are also available with the NetCDF main installation, in the directory \texttt{examples}.

  \item For more information about how to install NetCDF in your personal computer, from scratch, check Section \ref{ap:netcdf}.

\end{itemize}

\end{frame}

\begin{frame}[fragile]{File Reference: \texttt{simple\_xy\_wr.c}}

\begin{itemize}
\setlength\itemsep{0.4cm}

  \item This is an example program demonstrating a simple 2D write. It is intended to illustrate the use of the NetCDF C API.

    \begin{itemize}
    \setlength\itemsep{0.2cm}
      \item {\scriptsize \url{https://www.unidata.ucar.edu/software/netcdf/docs/simple__xy__wr_8c.html}}
      \item {\scriptsize \url{https://www.unidata.ucar.edu/software/netcdf/docs/simple__xy__wr_8c_source.html}}
    \end{itemize}

  \item Dependency graph for \verb|simple_xy_wr|:

\end{itemize}

\begin{center}
\includegraphics[scale=0.5]{fig/simple__xy__wr_8c__incl}
\end{center}

\end{frame}

\subsection{Code Analysis}

\begin{frame}[fragile]{File \texttt{simple\_xy\_wr.c}: Header and Constants Declaration}

\begin{columns}

\begin{column}{0.5\textwidth}

{\tiny

\begin{verbatim}

#include <stdlib.h>
#include <stdio.h>
#include <netcdf.h>

/* This is the name of the data file we will create. */
#define FILE_NAME "simple_xy.nc"

/* We are writing 2D data, a 6 x 12 grid. */
#define NDIMS 2
#define NX 6
#define NY 12

/* Handle errors by printing an error message and exiting with a
* non-zero status. */
#define ERRCODE 2
#define ERR(e) {printf("Error: %s\n", nc_strerror(e)); exit(ERRCODE);}

\end{verbatim}

}

\end{column}

\begin{column}{0.5\textwidth}

%\vspace*{-2cm}

{\footnotesize

\begin{itemize}

\item Standard C libraries and main NetCDF library.\\[0.6cm]

\item Define the name for the NetCDF file.\\[0.4cm]

\item Define the total number of dimensions.
\item Define each of the dimensions.\\[0.6cm]

\item Define error codes and messages.

\end{itemize}

}

\end{column}

\end{columns}

\end{frame}

\begin{frame}[fragile]{File \texttt{simple\_xy\_wr.c}: Variables Declaration}

\begin{columns}

\begin{column}{0.5\textwidth}

{\tiny

\begin{verbatim}

...
...
...

int main()
{
  /* When we create NetCDF variables and dimensions, we get back an
   * ID for each one. */
  int ncid, x_dimid, y_dimid, varid;
  int dimids[NDIMS];

  /* This is the data array we will write. It will be filled with a
   * progression of numbers for this example. */
  int data_out[NX][NY];

  /* Loop indexes, and error handling. */
  int x, y, retval;

  ...
  ...
  ...
}

\end{verbatim}

}

\end{column}

\begin{column}{0.5\textwidth}

\vspace*{-1cm}

{\footnotesize

\begin{itemize}
\setlength\itemsep{0.1cm}

\item Main program.

\item Note that each variable represents an id.

  \begin{itemize}
  {\tiny
    \item \verb|ncid|: id for the NetCDF file.
    \item \verb|x_dimid|: id for the x dimension.
    \item \verb|x_dimid|: id for the y dimension.
    \item \verb|dimids|: vector with all dimensions ids.\\
  }
  \end{itemize}

\item Vector that will store the data.

\end{itemize}

}

\end{column}

\end{columns}

\end{frame}

\begin{frame}[fragile]{File \texttt{simple\_xy\_wr.c}: Creating (loading!) Data}

\begin{columns}

\begin{column}{0.5\textwidth}

{\tiny

\begin{verbatim}

...

int main()
{
  ...

  /* Create some pretend data. If this wasn't an example program, we
   * would have some real data to write, for example, model
   * output. */
  for (x = 0; x < NX; x++)
     for (y = 0; y < NY; y++)
        data_out[x][y] = x * NY + y;

  ...
}

\end{verbatim}

}

\end{column}

\begin{column}{0.5\textwidth}

{\footnotesize

\begin{itemize}
\setlength\itemsep{0.5cm}

\item Real data can be loaded using different databases and functions in C.

\item In this example, the data is generated by a simple formula.

\end{itemize}

}

\end{column}

\end{columns}

\end{frame}

\begin{frame}[fragile]{File \texttt{simple\_xy\_wr.c}: Creating the NetCDF file}

\begin{columns}

\begin{column}{0.5\textwidth}

{\tiny

\begin{verbatim}

...

int main()
{
  ...

  /* Always check the return code of every NetCDF function call. In
   * this example program, any retval which is not equal to NC_NOERR
   * (0) will cause the program to print an error message and exit
   * with a non-zero return code. */

  /* Create the file. The NC_CLOBBER parameter tells NetCDF to
   * overwrite this file, if it already exists.*/
  if ((retval = nc_create(FILE_NAME, NC_CLOBBER, &ncid)))
     ERR(retval);

  ...
}

\end{verbatim}

}

\end{column}

\begin{column}{0.5\textwidth}

{\footnotesize

\begin{itemize}
\setlength\itemsep{0.5cm}

\item The function to create a NetCDF file is called \verb|nc_create|.

\item This function has three parameters:

  \begin{itemize}
  {\tiny
    \item The name of the file.
    \item The mode to open the file.
    \item It returns the id for the NetCDF file.\\
  }
  \end{itemize}

\end{itemize}

}

\end{column}

\end{columns}

\end{frame}

\begin{frame}[fragile]{File \texttt{simple\_xy\_wr.c}: Defining the Dimensions}

\begin{columns}

\begin{column}{0.5\textwidth}

{\tiny

\begin{verbatim}

...

int main()
{
  ...

  /* Define the dimensions. NetCDF will hand back an ID for each. */
  if ((retval = nc_def_dim(ncid, "x", NX, &x_dimid)))
     ERR(retval);
  if ((retval = nc_def_dim(ncid, "y", NY, &y_dimid)))
     ERR(retval);

  /* The dimids array is used to pass the IDs of the dimensions of
   * the variable. */
  dimids[0] = x_dimid;
  dimids[1] = y_dimid;

  ...
}

\end{verbatim}

}

\end{column}

\begin{column}{0.5\textwidth}

{\footnotesize

\begin{itemize}
\setlength\itemsep{0.5cm}

\item The function to define new dimensions in a NetCDF file is called \verb|nc_def_dim|.

\item This function has four parameters:

  \begin{itemize}
  {\tiny
    \item The id of the NetCDF file.
    \item The name of the dimension to be created.
    \item The size of the dimension to be created.
    \item It returns the id for created dimension.\\
  }
  \end{itemize}

\item The vector \verb|dimids| stores the ids for the created dimensions.

\end{itemize}

}

\end{column}

\end{columns}

\end{frame}

\begin{frame}[fragile]{File \texttt{simple\_xy\_wr.c}: Defining a Variable}

\begin{columns}

\begin{column}{0.5\textwidth}

{\tiny

\begin{verbatim}

...

int main()
{
  ...

  /* Define the variable. The type of the variable in this case is
   * NC_INT (4-byte integer). */
  if ((retval = nc_def_var(ncid, "data", NC_INT, NDIMS,
               dimids, &varid)))
     ERR(retval);

  ...
}

\end{verbatim}

}

\end{column}

\begin{column}{0.5\textwidth}

{\footnotesize

\begin{itemize}
\setlength\itemsep{0.5cm}

\item The function to define new variables in a NetCDF file is called \verb|nc_def_var|.

\item This function has six parameters:

  \begin{itemize}
  {\tiny
    \item The id of the NetCDF file.
    \item The name of the variable to be created.
    \item The type of the variable to be created.
    \item The number of dimensions of the variable.
    \item The vector that stores the ids for the dimensions.
    \item It returns the id for created variable.\\
  }
  \end{itemize}

\end{itemize}

}

\end{column}

\end{columns}

\end{frame}

\begin{frame}[fragile]{File \texttt{simple\_xy\_wr.c}: Defining a Variable}

\begin{columns}

\begin{column}{0.5\textwidth}

{\tiny

\begin{verbatim}

...

int main()
{
  ...

  /* End define mode. This tells NetCDF we are done defining
   * metadata. */
  if ((retval = nc_enddef(ncid)))
     ERR(retval);

  ...
}

\end{verbatim}

}

\end{column}

\begin{column}{0.5\textwidth}

{\footnotesize

\begin{itemize}
\setlength\itemsep{0.5cm}

\item Classic NetCDF:
\begin{itemize}
{\tiny
  \item In \textbf{define mode}, dimensions, variables, and new attributes can be created but variable data cannot be read or written.
  \item In \textbf{data mode}, data can be read or written and attributes can be changed, but new dimensions, variables, and attributes cannot be created.\\
}
\end{itemize}

\item \textbf{NOTE:} NetCDF-4 does not distinguish between define and data modes.

\end{itemize}

}

\end{column}

\end{columns}

\end{frame}

\begin{frame}[fragile]{File \texttt{simple\_xy\_wr.c}: Writing Data into the File}

\begin{columns}

\begin{column}{0.5\textwidth}

{\tiny

\begin{verbatim}

...

int main()
{
  ...

  /* Write the pretend data to the file. Although NetCDF supports
   * reading and writing subsets of data, in this case we write all
   * the data in one operation. */
  if ((retval = nc_put_var_int(ncid, varid, &data_out[0][0])))
     ERR(retval);

  ...
}

\end{verbatim}

}

\end{column}

\begin{column}{0.5\textwidth}

{\footnotesize

\begin{itemize}
\setlength\itemsep{0.5cm}

\item The function to write the data in a variable is called \verb|nc_put_var_*|. In this example, we have \verb|nc_put_var_int|.

\item This function has four parameters:

  \begin{itemize}
  {\tiny
    \item The id of the NetCDF file.
    \item The id of the variable that will store the data.
    \item (A pointer to) the data.\\
  }
  \end{itemize}


\end{itemize}

}

\end{column}

\end{columns}

\end{frame}

\begin{frame}[fragile]{File \texttt{simple\_xy\_wr.c}: Writing Data into the File}

\begin{columns}

\begin{column}{0.5\textwidth}

{\tiny

\begin{verbatim}

...

int main()
{
  ...

  /* Close the file. This frees up any internal NetCDF resources
   * associated with the file, and flushes any buffers. */
  if ((retval = nc_close(ncid)))
     ERR(retval);

  ...
}

\end{verbatim}

}

\end{column}

\begin{column}{0.5\textwidth}

{\footnotesize

\begin{itemize}
\setlength\itemsep{0.5cm}

\item The function to close a NetCDF file is called \verb|nc_close|.

\item This function has one parameter:

  \begin{itemize}
  {\tiny
    \item The id of the NetCDF file.
  }
  \end{itemize}


\end{itemize}

}

\end{column}

\end{columns}

\end{frame}

\begin{frame}[fragile]{File \texttt{simple\_xy\_wr.c}: Getting SUCCESS!}

\begin{columns}

\begin{column}{0.5\textwidth}

{\tiny

\begin{verbatim}

...

int main()
{
  ...

  printf("*** SUCCESS writing example file simple_xy.nc!\n");
  return 0;
}

\end{verbatim}

}

\end{column}

\begin{column}{0.5\textwidth}

{\footnotesize

\begin{itemize}
\setlength\itemsep{0.5cm}

\item If everything is done properly, we end the main program with a nice and encouraging message.

\item Hopefully, also with a new NetCDF file!

\end{itemize}

}

\end{column}

\end{columns}

\end{frame}

\subsection{Working and Analysing the NetCDF File}

\begin{frame}[fragile]{Using \texttt{nc-config}}

\begin{itemize}
\setlength\itemsep{0.5cm}

\item The \verb|nc-config| command-line utility assists with the setting of compiler and linker flags for building applications.

\item \verb|nc-config| is a simple script that reports the configuration flags used during the NetCDF build, as well as the installed version of the NetCDF C-based libraries.

\item It has lots of options, listed by invoking \verb|$(nc-config --all)|.

\item Here we will use \verb|nc-config| to compile and link a C application:

\begin{itemize}
\setlength\itemsep{0.4cm}
\item \verb|gcc myapp.c -o myapp $(nc-config --libs --cflags)|
\end{itemize}

\end{itemize}

\end{frame}

\begin{frame}[fragile]{Compiling and Running the File \texttt{simple\_xy\_wr.c}}

\begin{itemize}
\setlength\itemsep{0.6cm}

  \item Create (copy!) and compile the file \verb|simple_xy_wr.c|.

    \begin{itemize}
      \item {\footnotesize \verb|gcc simple_xy_wr.c -o simple_xy_wr $(nc-config --libs --cflags)|}
    \end{itemize}

  \item Run the file \verb|simple_xy_wr|.

      \begin{itemize}
      \setlength\itemsep{0.2cm}
        \item {\footnotesize \verb|./simple_xy_wr|}
        \item {\footnotesize \verb|*** SUCCESS writing example file simple_xy.nc!|}
      \end{itemize}

  \item Check that the file \verb|simple_xy.nc| is in your directory.

\end{itemize}

\end{frame}

\begin{frame}[fragile]{File Reference: \texttt{simple\_xy\_rd.c}}

\begin{itemize}
\setlength\itemsep{0.4cm}

  \item This is a simple example which reads a small dummy array that was written by \verb|simple_xy_wr.c|.

    \begin{itemize}
    \setlength\itemsep{0.2cm}
      \item {\scriptsize \url{https://www.unidata.ucar.edu/software/netcdf/docs/simple__xy__rd_8c.html}}
      \item {\scriptsize \url{https://www.unidata.ucar.edu/software/netcdf/docs/simple__xy__rd_8c_source.html}}
    \end{itemize}

  \item Dependency graph for \verb|simple_xy_rd|:

\end{itemize}

\begin{center}
\includegraphics[scale=0.5]{fig/simple__xy__rd_8c__incl}
\end{center}

\end{frame}

\begin{frame}[fragile]{File \texttt{simple\_xy\_rd.c}}

\begin{columns}

\begin{column}{0.5\textwidth}

{\tiny

\begin{verbatim}

int main()
{
   /* Open the file. NC_NOWRITE tells NetCDF we want read-only access
    * to the file.*/
   if ((retval = nc_open(FILE_NAME, NC_NOWRITE, &ncid)))
      ERR(retval);

   /* Get the varid of the data variable, based on its name. */
   if ((retval = nc_inq_varid(ncid, "data", &varid)))
      ERR(retval);

   /* Read the data. */
   if ((retval = nc_get_var_int(ncid, varid, &data_in[0][0])))
      ERR(retval);

   /* Check the data. */
   for (x = 0; x < NX; x++)
      for (y = 0; y < NY; y++)
         if (data_in[x][y] != x * NY + y)
            return ERRCODE;

   /* Close the file, freeing all resources. */
   if ((retval = nc_close(ncid)))
      ERR(retval);
}

\end{verbatim}

}

\end{column}

\begin{column}{0.5\textwidth}

{\footnotesize

\begin{itemize}
\setlength\itemsep{0.3cm}

\item The function to open a NetCDF file is called \verb|nc_open|. It is similar to the function \verb|nc_create| and it has the same parameters.

\item The function \verb|nc_inq_varid| is called to find the id of a variable. It has three parameters:

  \begin{itemize}
  {\tiny
    \item The id of the NetCDF file.
    \item The name of the variable.
    \item It returns the id for the variable.\\
  }
  \end{itemize}

\item The function \verb|nc_get_var_*| is called to read data of a variable. In this example, we have \verb|nc_get_var_int|. It has three parameters:

  \begin{itemize}
  {\tiny
    \item The id of the NetCDF file.
    \item The id of the variable.
    \item It returns the data stored in the variable.\\
  }
  \end{itemize}

\end{itemize}

}

\end{column}

\end{columns}

\end{frame}

\begin{frame}[fragile]{Reading the File \texttt{simple\_xy.nc}}

\begin{itemize}
\setlength\itemsep{0.6cm}

  \item Check that the file \verb|simple_xy.nc| is in your directory.

  \item Create (copy!), compile and run the file \verb|simple_xy_rd.c|.

        \begin{itemize}
          \item {\footnotesize  \verb|gcc simple_xy_rd.c -o simple_xy_rd $(nc-config --libs --cflags)| }
        \end{itemize}

  \item Run the file \verb|simple_xy_rd|.

        \begin{itemize}
        \setlength\itemsep{0.2cm}
          \item {\footnotesize  \verb|./simple_xy_rd|}
          \item {\footnotesize  \verb|*** SUCCESS reading example file simple_xy.nc!|}
        \end{itemize}

\end{itemize}

\end{frame}

\section{NetCDF Utilities}

\begin{frame}[fragile]{\texttt{ncdump} and \texttt{ncgen}}

\begin{itemize}
\setlength\itemsep{1cm}

  \item \texttt{ncdump} and \texttt{ncgen} are inverses:\\[0.4cm]

  \begin{center}
  \includegraphics[scale=0.5]{fig/nc-inv}
  \end{center}

  \item Used together, \texttt{ncdump} and \texttt{ncgen} can accomplish simple NetCDF manipulations with little or no programming.

\end{itemize}

\end{frame}

\begin{frame}[fragile]{Editing a NetCDF File}

\begin{itemize}
\setlength\itemsep{0.4cm}
  \item To edit metadata or data in a NetCDF file:\\[0.4cm]

\begin{center}
\includegraphics[scale=0.5]{fig/nc-edit}
\end{center}

  \begin{itemize}
  \setlength\itemsep{0.3cm}
    \item Use \texttt{ncdump} to convert NetCDF file to CDL.
    \item Use a text editor to make desired change to CDL.
    \item Use \texttt{ncgen} to turn modified CDL back into NetCDF file.
    \item \textbf{Note:} This option is not practical for huge NetCDF files or if one intend to modify lots of files. For that, need to write a program using NetCDF library.
  \end{itemize}

\end{itemize}

\end{frame}

\begin{frame}[fragile]{Creating a NetCDF File}

\begin{itemize}
  \item To create a new NetCDF file with lots of metadata:\\[0.3cm]

\begin{center}
\includegraphics[scale=0.5]{fig/nc-creat}
\end{center}

    \begin{itemize}
    \setlength\itemsep{0.1cm}
      \item Use a text editor to write a CDL file with lots of metadata but little or no data.
      \item Use \texttt{ncgen} to generate corresponding C or Fortran program for writing NetCDF.
      \item Insert appropriate NetCDF \textbf{var\_put} calls for writing data.
      \item Compile and run program to create NetCDF file.
      \item Use \texttt{ncdump} to verify result.
    \end{itemize}

\end{itemize}

\end{frame}

\begin{frame}[fragile]{Using \texttt{ncdump}}

\begin{itemize}
\setlength\itemsep{0.5cm}

  \item Inspect the file \verb|simple_xy.nc| using \texttt{ncdump}:\\[0.4cm]

  \begin{itemize}
  \setlength\itemsep{0.5cm}

    \item \verb|ncdump simple_xy.nc|

  \end{itemize}

  \item Inspect the metadata of the file \verb|simple_xy.nc| using \texttt{ncdump}:\\[0.4cm]

  \begin{itemize}
  \setlength\itemsep{0.5cm}

    \item \verb|ncdump -h simple_xy.nc|

  \end{itemize}

  \item Check other options for \texttt{ncdump} with:\\[0.4cm]

  \begin{itemize}
  \setlength\itemsep{0.5cm}

    \item \verb|ncdump --help|

  \end{itemize}

\end{itemize}

\end{frame}

\begin{frame}[fragile]{NetCDF CDL Format}

\begin{figure}
\centering
\begin{varwidth}{\linewidth}
{\footnotesize

\begin{verbatim}

netcdf simple_xy {
dimensions:
	x = 6 ;
	y = 12 ;
variables:
	int data(x, y) ;
data:

 data =
  0, 1, 2, 3, 4, 5, 6, 7, 8, 9, 10, 11,
  12, 13, 14, 15, 16, 17, 18, 19, 20, 21, 22, 23,
  24, 25, 26, 27, 28, 29, 30, 31, 32, 33, 34, 35,
  36, 37, 38, 39, 40, 41, 42, 43, 44, 45, 46, 47,
  48, 49, 50, 51, 52, 53, 54, 55, 56, 57, 58, 59,
  60, 61, 62, 63, 64, 65, 66, 67, 68, 69, 70, 71 ;
}

\end{verbatim}

}
\end{varwidth}
\end{figure}

\end{frame}

\begin{frame}[fragile]{Using \texttt{ncgen}}

\begin{itemize}

  \item Create a NetCDF file using \texttt{ncgen} and the CDL output\\[0.2cm]

    \begin{itemize}
    \setlength\itemsep{0.2cm}

      \item \verb|ncdump simple_xy.nc > simple_xy_test.cdl|
      \item \verb|more simple_xy_test.cdl|
      \item \verb|ncgen -b simple_xy_test.cdl|
      \item \verb|cmp simple_xy_test.nc simple_xy.nc|

    \end{itemize}

\end{itemize}

\begin{center}
\includegraphics[scale=0.4]{fig/c-ncgen}
\end{center}

\end{frame}

\begin{frame}[fragile]{Creating the C File}

\begin{itemize}
\setlength\itemsep{0.4cm}

  \item Create a C file using \texttt{ncgen} and the CDL output\\[0.4cm]

  \begin{itemize}
  \setlength\itemsep{0.5cm}

    \item \verb|ncgen -lc simple_xy_test.cdl > simple_xy_test.c|
    \item \verb|more simple_xy_test.c|
    \item What is the difference between the files \verb|simple_xy_test.c| and \verb|simple_xy_wr.c|?

    \begin{itemize}
      \item \verb|cmp simple_xy_test.c simple_xy_wr.c|
      \item \verb|meld simple_xy_test.c simple_xy_wr.c|
    \end{itemize}

  \end{itemize}

\end{itemize}

\end{frame}

\begin{frame}[fragile]{Starting All Over Again!}

\begin{itemize}
  \setlength\itemsep{0.5cm}

  \item \verb|gcc simple_xy_test.c -o simple_xy_test $(nc-config --libs --cflags)|\\
  \item \verb|mv simple_xy_test.nc simple_xy_test2.nc|
  \item \verb|./simple_xy_test|
  \item \verb|cmp simple_xy_test.nc simple_xy_test2.nc|

\end{itemize}

\end{frame}

\begin{frame}[fragile]{Using \texttt{ncview}}

\begin{columns}

\begin{column}{0.4\textwidth}

\begin{itemize}
  \item \verb|ncview simple_xy.nc|
\end{itemize}

\end{column}

\begin{column}{0.6\textwidth}
\begin{center}
\raisebox{0.5cm}{\includegraphics[scale=0.5]{fig/simple_xy}}
\end{center}
\end{column}

\end{columns}

\begin{figure}
\centering
\begin{varwidth}{\linewidth}

{\tiny

\begin{verbatim}

netcdf simple_xy {
dimensions:
	x = 6 ;
	y = 12 ;
variables:
	int data(x, y) ;
data:

 data =
  0, 1, 2, 3, 4, 5, 6, 7, 8, 9, 10, 11,
  12, 13, 14, 15, 16, 17, 18, 19, 20, 21, 22, 23,
  24, 25, 26, 27, 28, 29, 30, 31, 32, 33, 34, 35,
  36, 37, 38, 39, 40, 41, 42, 43, 44, 45, 46, 47,
  48, 49, 50, 51, 52, 53, 54, 55, 56, 57, 58, 59,
  60, 61, 62, 63, 64, 65, 66, 67, 68, 69, 70, 71 ;
}

\end{verbatim}

}

\end{varwidth}
\end{figure}

\end{frame}

\begin{frame}[fragile]{Using \texttt{ncview} -- A Global Example}

\begin{center}
\includegraphics[scale=0.4]{fig/elevations} \\
File \verb|elevations.nc|\\[0.3cm]
\end{center}

\end{frame}

\begin{frame}[fragile]{Global Potential Vegetation Dataset}

\begin{itemize}
\setlength\itemsep{0.4cm}

  \item File \verb|vegtype_5min.nc| -- NetCDF 5 min data
  \item File \verb|vegtype_0.5.nc| -- NetCDF data aggregated to a 0.5 deg resolution

\end{itemize}

\begin{center}
\begin{tabular}{cc}
\raisebox{1cm}{\includegraphics[scale=0.4]{fig/veg-5min}} &
\includegraphics[scale=0.4]{fig/veg-05min} \\
File \verb|vegtype_5min.nc| & File \verb|vegtype_0.5.nc|
\end{tabular}
\end{center}

{\tiny Files available at \url{http://nelson.wisc.edu/sage/data-and-models/global-potential-vegetation/index.php}}

\end{frame}

\begin{frame}[fragile]{Northern Hemisphere EASE-Grid Weekly Snow Cover}

\begin{itemize}

  \item File \verb|snowcover.mon.mean.nc|

\end{itemize}

\begin{center}
{\tiny
\begin{tabular}{ccc}
\includegraphics[scale=0.35]{fig/snow12} &
\includegraphics[scale=0.35]{fig/snow11} &
\includegraphics[scale=0.35]{fig/snow10} \\
January & February & March \\ & & \\
\includegraphics[scale=0.35]{fig/snow9} &
\includegraphics[scale=0.35]{fig/snow8} &
\includegraphics[scale=0.35]{fig/snow7} \\
April & May & June \\ & & \\
\includegraphics[scale=0.35]{fig/snow6} &
\includegraphics[scale=0.35]{fig/snow5} &
\includegraphics[scale=0.35]{fig/snow4} \\
July & August & September \\ & & \\
\includegraphics[scale=0.35]{fig/snow3} &
\includegraphics[scale=0.35]{fig/snow2} &
\includegraphics[scale=0.35]{fig/snow1} \\
October & November & December
\end{tabular}
}
\end{center}

{\tiny File available at \url{https://psl.noaa.gov/data/gridded/data.snowcover.html}}.

\end{frame}

\begin{frame}[fragile]{Other NetCDF Utilities}

\begin{itemize}
\setlength\itemsep{0.5cm}

  \item Many other useful netCDF utilities developed by third parties are available:

  \begin{itemize}
    \item {\footnotesize NCAR Command Language (NCL)}
    \begin{itemize}
      \item {\scriptsize \url{https://www.unidata.ucar.edu/software/netcdf/software.html#NCL}}
    \end{itemize}
    \item {\footnotesize NCO (NetCDF operators)}
    \begin{itemize}
      \item {\scriptsize \url{https://www.unidata.ucar.edu/software/netcdf/software.html#NCO}}
    \end{itemize}
    \item {\footnotesize CDO (Climate Data Operators)}
    \begin{itemize}
      \item {\scriptsize \url{https://www.unidata.ucar.edu/software/netcdf/software.html#CDO}}
    \end{itemize}
  \end{itemize}

  \item For additional utility software, consult:

  \begin{itemize}
    \item {\footnotesize Unidata's Software for Manipulating or Displaying NetCDF Data}
    \begin{itemize}
      \item {\scriptsize \url{http://www.unidata.ucar.edu/netcdf/software.html}}
    \end{itemize}
  \end{itemize}

\end{itemize}

\end{frame}

\begin{comment}

\section{NetCDF Operators}

\begin{frame}[fragile]{NetCDF Operators (NCO)\lr{Is there time to explore?}}

\begin{itemize}

  \item NCO is a package of command line operators that manipulates generic NetCDF data and supports some CF conventions. The NCO utilities are:

{ \footnotesize

\begin{table}
\begin{center}
\begin{tabular}{cc}

\begin{tabular}{|l|l|}
\hline
ncap2 & arithmetic processor \\ \hline
ncatted & attribute editor \\ \hline
ncbo & binary operator \\ \hline
ncdiff & differencer \\ \hline
ncea & ensemble averager \\ \hline
ncecat & ensemble concatenator \\ \hline
ncflint & file interpolator \\ \hline
\end{tabular}

&

\begin{tabular}{|l|l|}
\hline
ncks & kitchen sink \\
& (extract, cut, paste, print data) \\ \hline
ncpdq & permute dimensions quickly \\ \hline
ncra & running averager \\ \hline
ncrcat & record concatenator \\ \hline
ncrename & renamer \\ \hline
ncwa & weighted averager \\ \hline
\end{tabular}

\end{tabular}
\end{center}
\end{table}

}

\item NCO utilities have as a goal being as generic as possible, imposing no limitations on data dimensionality, size, or type. All established national and international climate modelling centers now install and maintain NCO for their users for data post-processing, hyper-slabbing, and serving.

\end{itemize}

\end{frame}

\begin{frame}[fragile]{Collective and Independent Operations with Parallel I/O}

\begin{itemize}

  \item You must build NetCDF-4 properly to take advantage of parallel features.

  \begin{itemize}
  \setlength\itemsep{0.3cm}

    \item HDF5 must be built with \verb|--enable-parallel|.

    \item Typically the CC environment variable is set to mpicc, and the FC to mpifc (or some local variant.) You must build HDF5 and NetCDF-4 with the same compiler and compiler options.

    \item The NetCDF configure script will detect the parallel capability of HDF5 and build the NetCDF-4 parallel I/O features automatically. No configure options to the netcdf configure are required.

    \item For parallel builds you must include \verb|"netcdf_par.h"| before (or instead of) \verb|netcdf.h|.

    \item Parallel tests are run by shell scripts. For testing to work on your platform the \verb|mpiexec| utility must be supported.

    \item Parallel I/O is tested in \verb|nc_test4/tst_parallel.c|, \verb|nc_test4/tst_parallel2.c|, and \verb|nc_test4/tst_parallel3.c|, and other programs, if the \verb|--enable-parallel-tests| option to configure is used.

  \end{itemize}

\end{itemize}

\end{frame}

\section{Parallel I/O}

\begin{frame}[fragile]{Parallel I/O Example -- MPI-IO}

\begin{figure}
\centering
\begin{varwidth}{\linewidth}
{ \tiny

\begin{verbatim}

#include <stdio.h>
#include <mpi.h>
int main(int argc, char *argv[])
{
 MPI_File fh;
 int buf[1000], rank;
 MPI_Init(0,0);
 MPI_Comm_rank(MPI_COMM_WORLD, &rank);
 MPI_File_open(MPI_COMM_WORLD, "test.out",
 MPI_MODE_CREATE|MPI_MODE_WRONLY,
 MPI_INFO_NULL, &fh);
 if (rank == 0)
 MPI_File_write(fh, buf, 1000, MPI_INT, MPI_STATUS_IGNORE);
 MPI_File_close(&fh);
 MPI_Finalize();
 return 0;
}

\end{verbatim}

}
\end{varwidth}
\end{figure}

\end{frame}

\begin{frame}[fragile]{Parallel I/O Example -- NetCDF4 -- Part I}

\begin{figure}
\centering
\begin{varwidth}{\linewidth}
{ \tiny

\begin{verbatim}

/* Initialize MPI. */
    MPI_Init(&argc,&argv);
    MPI_Comm_size(MPI_COMM_WORLD, &mpi_size);
    MPI_Comm_rank(MPI_COMM_WORLD, &mpi_rank);
    MPI_Get_processor_name(mpi_name, &mpi_namelen);

    if (mpi_rank == 1)
       printf("\n*** tst_parallel testing very basic parallel access.\n");

    /* Create a parallel netcdf-4 file. */
    if ((res = nc_create_par(FILE, NC_NETCDF4|NC_MPIIO, comm,
			     info, &ncid))) ERR;

    /* Create two dimensions. */
    if ((res = nc_def_dim(ncid, "d1", DIMSIZE, dimids))) ERR;
    if ((res = nc_def_dim(ncid, "d2", DIMSIZE, &dimids[1]))) ERR;

    /* Create one var. */
    if ((res = nc_def_var(ncid, "v1", NC_INT, NDIMS, dimids, &v1id))) ERR;

    if ((res = nc_enddef(ncid))) ERR;

\end{verbatim}

}
\end{varwidth}
\end{figure}

\end{frame}

\begin{frame}[fragile]{Parallel I/O Example -- NetCDF4 -- Part II}

\begin{figure}
\centering
\begin{varwidth}{\linewidth}
{ \tiny

\begin{verbatim}

    /* Set up slab for this process. */
    start[0] = mpi_rank * DIMSIZE/mpi_size;
    start[1] = 0;
    count[0] = DIMSIZE/mpi_size;
    count[1] = DIMSIZE;

    /* Create phoney data. We're going to write a 24x24 array of ints,
       in 4 sets of 144. */
    for (i=mpi_rank*QTR_DATA; i < (mpi_rank+1)*QTR_DATA; i++)
       data[i] = mpi_rank;

    /*if ((res = nc_var_par_access(ncid, v1id, NC_COLLECTIVE)))
      ERR;*/
    if ((res = nc_var_par_access(ncid, v1id, NC_INDEPENDENT))) ERR;

    /* Write slabs of phoney data. */
    if ((res = nc_put_vara_int(ncid, v1id, start, count,
			       &data[mpi_rank*QTR_DATA]))) ERR;

    /* Close the netcdf file. */
    if ((res = nc_close(ncid)))	ERR;

    /* Shut down MPI. */
    MPI_Finalize();

    return 0;

\end{verbatim}

}
\end{varwidth}
\end{figure}

\end{frame}

\end{comment}

\section{Practising}

\begin{frame}[fragile]{Files for Practising}

\begin{itemize}
\setlength\itemsep{0.4cm}

  \item File \verb|simple_xy_nc4|\\[0.3cm]
    \begin{itemize}
    \setlength\itemsep{0.2cm}
      \item {\footnotesize Write/Read the \verb|simple_xy| file with some of the features of NetCDF-4.}
      \item {\scriptsize \url{https://www.unidata.ucar.edu/software/netcdf/docs/simple__xy__nc4__wr_8c.html}}
      \item {\scriptsize \url{https://www.unidata.ucar.edu/software/netcdf/docs/simple__xy__nc4__rd_8c.html}}
    \end{itemize}

  \item File \verb|simple_nc4|\\[0.3cm]
    \begin{itemize}
    \setlength\itemsep{0.2cm}
      \item {\footnotesize Write/Read a file demonstrating some of the features of NetCDF-4.}
      \item {\scriptsize \url{https://www.unidata.ucar.edu/software/netcdf/docs/simple__nc4__wr_8c.html}}
      \item {\scriptsize \url{https://www.unidata.ucar.edu/software/netcdf/docs/simple__nc4__rd_8c.html}}
    \end{itemize}

\end{itemize}

\end{frame}

\begin{frame}[fragile]{Files for Practising}

\begin{itemize}
\setlength\itemsep{0.4cm}

  \item File \verb|sfc_pres_temp|\\[0.3cm]
    \begin{itemize}
    \setlength\itemsep{0.2cm}
      \item {\footnotesize This is an example program which writes/reads surface pressure and temperatures.}
      \item {\scriptsize \url{https://www.unidata.ucar.edu/software/netcdf/docs/sfc__pres__temp__wr_8c.html}}
      \item {\scriptsize \url{https://www.unidata.ucar.edu/software/netcdf/docs/sfc__pres__temp__rd_8c.html}}
    \end{itemize}

  \item File \verb|pres_temp_4D|\\[0.3cm]
    \begin{itemize}
    \setlength\itemsep{0.2cm}
      \item {\footnotesize This is an example program which writes/reads 4D pressure and temperatures.}
      \item {\scriptsize \url{https://www.unidata.ucar.edu/software/netcdf/docs/pres__temp__4D__wr_8c.html}}
      \item {\scriptsize \url{https://www.unidata.ucar.edu/software/netcdf/docs/pres__temp__4D__rd_8c.html}}
    \end{itemize}

\end{itemize}

\end{frame}

\begin{frame}[fragile]{Files for Practising}

Global Potential Vegetation Dataset

    \begin{itemize}
    \setlength\itemsep{0.2cm}
      \item File \verb|vegtype_5min.nc| -- NetCDF 5 min data
      \item File \verb|vegtype_0.5.nc| -- NetCDF data aggregated to a 0.5 deg resolution
      \item Files available at:
        \begin{itemize}
          \item {\scriptsize \url{http://nelson.wisc.edu/sage/data-and-models/global-potential-vegetation/index.php}}
        \end{itemize}
    \end{itemize}

\vspace*{0.3cm}

Northern Hemisphere EASE-Grid Weekly Snow Cover

    \begin{itemize}
    \setlength\itemsep{0.2cm}
      \item File \verb|snowcover.mon.mean.nc| -- Monthly Mean
      \item File \verb|snowcover.mon.ltm.nc|	-- Monthly Long Term Mean
      \item Files available at:
        \begin{itemize}
          \item {\scriptsize \url{https://psl.noaa.gov/data/gridded/data.snowcover.html}}
        \end{itemize}
    \end{itemize}

\end{frame}

\begin{frame}[fragile]{Summary of Actions}

\begin{itemize}
\setlength\itemsep{0.3cm}

  \item Inspect the write and read files in C code.
  \item Compile and run the write/read C files.
  \item Inspect the output NetCDF file (.nc) using \texttt{ncdump}.
  \item Create a CDL file for the NetCDF file.
  \item Recreate the NetCDF file using \texttt{ncgen} and the CDL file.
  \item Recreate the C file using \texttt{ncgen} and the CDL file.
  \item Visualise the data in the NetCDF file with \texttt{ncview}.
  \item Change dimensions, variables, and attributes and rebuild the previous steps.

\end{itemize}

\end{frame}

\appendix

\begin{frame}[fragile]{}

{ \huge \color{EsiBlue}{ Appendix}}

\end{frame}

\section{Starting with NetCDF}
\label{ap:netcdf}

\subsection{Building NetCDF}

\begin{frame}[fragile]{Building NetCDF from Scratch}

\begin{itemize}
\setlength\itemsep{0.8cm}

  \item The usual way of building NetCDF requires the HDF5, zlib, curl and m4 libraries.

  \item Files for the libraries can be found in:

  \begin{center}
  \url{ftp://ftp.unidata.ucar.edu/pub/netcdf/netcdf-4}
  \end{center}

  \item The following slides presents the steps for installing NetCDF in Ubuntu 18.04 and 20.04 for a user named \textbf{username}. Adapt the path to your own user.

\end{itemize}

\end{frame}

\subsection{Installing curl and m4}

\begin{frame}[fragile]{Installing curl and m4}

\begin{itemize}
\setlength\itemsep{0.6cm}

  \item apt-get install libcurl4-openssl-dev
  \item apt-get install m4

\end{itemize}

\end{frame}

\subsection{Installing zlib}

\begin{frame}[fragile]{Installing zlib}

\begin{itemize}
\setlength\itemsep{0.3cm}

  \item {\footnotesize wget \url{ftp://ftp.unidata.ucar.edu/pub/netcdf/netcdf-4/zlib-1.2.8.tar.gz}}
  \begin{itemize}
    \item Newest version to later use \texttt{ncview}
    \item {\scriptsize wget \url{https://sourceforge.net/projects/libpng/files/zlib/1.2.9/zlib-1.2.9.tar.gz}}
  \end{itemize}
  \item tar -xvzf zlib-1.2.8.tar.gz
  \item cd zlib-1.2.8
  \item mkdir /home/username/local/
  \item ./configure {-}{-}prefix=/home/username/local/
  \item make check install

\end{itemize}

\end{frame}

\subsection{Installing HDF5}

\begin{frame}[fragile]{Installing HDF5}

\begin{itemize}
\setlength\itemsep{0.3cm}

  \item {\footnotesize wget \url{ftp://ftp.unidata.ucar.edu/pub/netcdf/netcdf-4/hdf5-1.8.13.tar.gz}}
  \item tar -xvzf hdf5-1.8.13.tar.gz
  \item cd hdf5-1.8.13
  \item ./configure {-}{-}with-zlib=/home/username/local/ {-}{-}prefix=/home/username/local/
  \item make
  \item make check
  \item make install
  \begin{itemize}
    \item make check install
    \item If not done separately, it might not work!
  \end{itemize}
\end{itemize}

\end{frame}

\subsection{Installing NetCDF}

\begin{frame}[fragile]{Installing NetCDF}

\begin{itemize}
\setlength\itemsep{0.3cm}

  \item Check the latest version at \url{https://www.unidata.ucar.edu/downloads/netcdf/}
  \item {\footnotesize wget \url{ftp://ftp.unidata.ucar.edu/pub/netcdf/netcdf-c-4.7.4.tar.gz}}
  \item tar -xvzf netcdf-c-4.7.4.tar.gz
  \item cd netcdf-c-4.7.4
  \item CPPFLAGS=-I/home/username/local/include LDFLAGS=-L/home/username/local/lib ./configure {-}{-}prefix=/home/username/local
  \item make check install

\end{itemize}

\end{frame}

\begin{frame}[fragile]{Finishing the Set Up}

\begin{itemize}
\setlength\itemsep{0.8cm}
  \item Link the NetCDF library\\
  \begin{itemize}
  \setlength\itemsep{0.2cm}
    \item export \verb|LD_LIBRARY_PATH=/home/username/local/lib/|
    \item \verb|sudo ldconfig|
  \end{itemize}

\end{itemize}

\end{frame}

\subsection{File \texttt{simple\_xy\_wr.c}}

% Inserting [fragile] to be able to work with \verb

\acknowledgement

\end{document}
